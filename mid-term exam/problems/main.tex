\documentclass{ctexart}
\author{}
\date{2016.11.17}
\title{mid-term exam}
\bibliographystyle{plain}
\newenvironment{keywords} {\\�ؼ��֣�\kaishu\zihao{-5}} {}
\usepackage{amsmath}
\usepackage{amssymb}
\usepackage[numbers,sort&compress,square,super]{natbib}
\usepackage{graphicx}
\usepackage{caption,subcaption}%����ͼ���ӱ���
\usepackage{asymptote}
%\usepackage{ctex}

\begin{document}
\maketitle

\begin{enumerate}
	\item assume some 2-dimensional flow field satisfy
	\begin{align*}
	u=x+t\\
	v=-y+t
	\end{align*}
	determine the streamline and pathline that is through point $(-1,-1)$,when $t=0$.
	
	\item assume some 2-dimensional flow field satisfy
	\begin{align*}
	u=x+t\\
	v=y+t
	\end{align*}
	and let $x=a,y=b$ when $t=0$, find the Lagrangian velocity expression.
	
	
%	\item
%	A proposed conservation law for $\xi$, a new fluid property, takes the following form:
%		\begin{align}
%			\frac{d}{dt}\int_{V(t)} \rho \xi dV+\int_{A(t)}\Theta \cdot \mathbf{n} ds=0
%		\end{align}
%	where $V(t)$ is a material volume that moves with the
%	fluid velocity $\mathbf{u}$, $A(t)$ is the surface of V(t), r is the fluid density, and $\Theta=-\rho \gamma \nabla \xi$\\
%	a) What partial differential equation is implied by the above conservation
%	statement?\\
%	b) Use the part a) result and the continuity equation to show: 
%	\begin{align}
%	\frac{\partial \xi}{\partial t}+\mathbf{u}\cdot \nabla \xi=\frac{1}{\rho}\nabla \cdot(\rho \gamma \nabla \xi)
%	\end{align}
	
	
	\item
	assume some flow in a tube is steady.the cross-section area,density,and velocity is $A(x),\rho(x),u(x)$ respectively,deduct the mass conservation equation.
	
	\item show mass conservation of  incompressible flow is
	\begin{align}
	\nabla \cdot \mathbf{u}=0
	\end{align}
	
	
\end{enumerate}


%\section{Mid-term exam}
%\begin{enumerate}
%	\item suppose a fluid element moves through the equation as follow,
%	\begin{align*}
%	x=2+0.01\sqrt{t^5}\\
%	y=2+0.01\sqrt{t^5}\\
%	z=2
%	\end{align*}
%	find the acceleration when the particle is at $x=8$
%	
%	\item
%	given a velocity field as follow,
%	\begin{align*}
%	u=yzt\\
%	v=zxt\\
%	w=0
%	\end{align*}
%	find the acceleration of fluid element at (2,5,3) when $t=10$
%	
%	
%	\item 
%	given a \textbf{incompressible} flow whose $x$ component of velocity is
%	\begin{align}
%	u=ax^2+by
%	\end{align}
%	where $a,b$ are constant, determine the $y$ component of velocity $v$. assume $y=0$ when $v=0$
%	
%	
%	\item
%	given a \textbf{incompressible} flow whose $x$ component of velocity is
%	\begin{align}
%	u=e^{-x}\cosh y +1
%	\end{align}
%	where $a,b$ are constant, determine the $y$ component of velocity $v$. assume $y=0$ when $v=0$
%	
%\end{enumerate}




\end{document}
