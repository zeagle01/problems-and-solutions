\documentclass[12pt]{article}
\usepackage{fancyhdr}
\usepackage{amsmath,amsfonts,enumerate}
\usepackage{color,graphicx}
\usepackage{bm}
\pagestyle{fancy}
%%%%%%%%%%%%%%%%%%%%%%%%%%%%%%%%%%%%%%%%%%%%%%%%%
% Do your customization here
%%%%%%%%%%%%%%%%%%%%%%%%%%%%%%%%%%%%%%%%%%%%%%%%%
\newcommand{\masunitnumber}{}
\newcommand{\examdate}{November 2016}
\newcommand{\academicyear}{2016-2017}
\newcommand{\semester}{I}
\newcommand{\coursename}{Transport Pheonomena}
\newcommand{\numberofhours}{2}

\newcommand{\ZZ}{\mathbb{Z}}
\newcommand{\CC}{\mathbb{C}}
\newcommand{\RR}{\mathbb{R}}
\newcommand{\FF}{\mathbb{F}}
%\DeclareMathOperator{\diam}{diam}
%%%%%%%%%%%%%%%%%%%%%%%%%%%%%%%%%%%%%%%%%%%%%%%%%
% Don't touch anything from here till instructions
% to candidates
%%%%%%%%%%%%%%%%%%%%%%%%%%%%%%%%%%%%%%%%%%%%%%%%%
\lhead{}
\rhead{}
\chead{{\bf SOUTHERN UNIVERSITY OF SCIENCE AND TECHNOLOGY}}
\lfoot{}
\rfoot{}
\cfoot{}
\begin{document}
\setlength{\headsep}{5truemm}
\setlength{\headheight}{14.5truemm}
\setlength{\voffset}{-0.45truein}
\renewcommand{\headrulewidth}{0.0pt}
\begin{center}
SEMESTER \semester\ EXAMINATION \academicyear
\end{center}
\begin{center}
{\bf \masunitnumber\ -- \coursename}
\end{center}
\vspace{20truemm}

\noindent \examdate\hspace{55truemm} TIME ALLOWED: \numberofhours\ HOURS

\vspace{19truemm}
\hrule
\vspace{19truemm}

\noindent\underline{INSTRUCTIONS TO CANDIDATES}
\vspace{8truemm}
%%%%%%%%%%%%%%%%%%%%%%%%%%%%%%%%%%%%%%%%%%%%%%%%%%%%%%
% Adjust your instructions here
%%%%%%%%%%%%%%%%%%%%%%%%%%%%%%%%%%%%%%%%%%%%%%%%%%%%%%
\begin{enumerate}
	\item This examination paper contains {\bf 8} questions.
	
	\item Answer all questions. 
	The marks for each question are indicated at the beginning of each question.
	
	
	\item This {\bf IS NOT an OPEN BOOK} exam.
	
	\item Candidates may use calculators. However, they should write down systematically the steps in the workings.
	
	\item Some formulas that might help.\\
	For laminar pipe flow,
	$u_{avg}=-\frac{1}{8\mu}\frac{\Delta p}{\Delta x}R^2$,$\Delta p$ is the pressure loss due to viscous effects, $R$ is the radius of pipe\\
	For pipe flow, $h_f=f\frac{L}{d}\frac{v^2}{2g}$, $f$ is the friction factor\\
	The acceleration of gravity $g=\mathrm{10m\cdot s^{-2}}$\\
\end{enumerate}


%%%%%%%%%%%%%%%%%%%%%%%%%%%%%%%%%%%%%%%%%%%%%%%%%
% leave this as it is
%%%%%%%%%%%%%%%%%%%%%%%%%%%%%%%%%%%%%%%%%%%%%%%%%
\newpage
\lhead{}
\rhead{\masunitnumber}
\chead{}
\lfoot{}
\cfoot{\thepage}
\rfoot{}
\setlength{\footskip}{45pt}
%%%%%%%%%%%%%%%%%%%%%%%%%%%%%%%%%%%%%%%%%%%%%%%%%%
% put your exam questions here
%%%%%%%%%%%%%%%%%%%%%%%%%%%%%%%%%%%%%%%%%%%%%%%%%%


\newpage
\paragraph{Problem 1.}\hfill (10 marks)\\
Determine the difference in pressure between the inside and outside of a soap film bubble at 20$^\circ\mathrm{C}$, if the diameter of the bubble is 5 mm. (The coefficient of surface tension $\sigma$=0.025 N/m.)

\bigskip
\paragraph{Problem 2.}\hfill (14 marks)\\
Gate $AB$ in figure below is semicircular, hinged at $B$, and held by a horizontal force $P$ at $A$. What force $P$ is required for equilibrium?
\begin{figure}[hb]
	\centering
	\includegraphics[width=0.5\linewidth]{"figs/problem 2B"}
	\caption{Problem 2}
	\label{fig:problem-2}
\end{figure}

\begin{figure}[h]
	\centering
	\includegraphics[width=0.5\linewidth]{"figs/problem 2B hint"}
	\caption{Centroidal coordinates and moments of area for a semicircle}
	\label{fig:problem-2 hint}
\end{figure}



\newpage
\paragraph{Problem 3.}\hfill (14 marks)\\
Assume some 2-dimensional flow field satisfy
\begin{align*}
u=x+t\\
v=y+t
\end{align*}
and let $x=a,y=b$ when $t=0$, find the Lagrangian velocity expression.



\bigskip
\paragraph{Problem 4.}\hfill (10 marks)\\
Show mass conservation of  incompressible flow is
\begin{align*}
\nabla \cdot \mathbf{u}=0
\end{align*}


\paragraph{Problem 5.}\hfill (12 marks)\\
Below is a picture which describes the siphon phenomenon: the water is siphoned from a large tank through a constant diameter hose. Assume water to be inviscid, incompressible and flow to be steady. (The acceleration of gravity $g=\mathrm{10 \, m\cdot s^{-2}}$.) Please determine:
\begin{enumerate}[(a)]
	\item
	velocity of water leaving (3) as a free jet
	\item
	water pressure in tube at (2)
\end{enumerate}
\begin{figure}[hb]
	\centering
	\includegraphics[width=0.5\linewidth]{"figs/problem 5"}
	\caption{Problem 5}
	\label{fig:problem-5}
\end{figure}


\newpage
\paragraph{Problem 6.}\hfill (14 marks)\\
In flow past a flat plate, the boundary layer thickness $\delta$ varies with distance $x$, freestream velocity $U$, viscosity $\mu$, and density $\rho$. Find the dimensionless parameters for this problem.



\paragraph{Problem 7.}\hfill (10 marks)\\
Write out the three components of $\frac{D\textbf{u}}{Dt}=-\frac{1}{\rho}\nabla p+\nu \nabla^2 \textbf{u}$ in x-y-z Cartesian coordinates.b) Set u =(u(y), 0, 0), and simplified the x- and y-momentum equations.\\



\paragraph{Problem 8.}\hfill (16 marks)\\
The tank–pipe system of Fig.xxx is to deliver at least $11\mathrm{m^3/h}$ of water at $20^{\circ}$C to the reservoir. What is the maximum roughness height  allowable for the pipe? ($\rho=1000\mathrm{kg/m^3} \text{ , } \mu=0.001\mathrm{kg/(m\cdot s)}$ and $g=\mathrm{10m\cdot s^{-2}}$)\\
hint:$h_f=f\frac{L}{d}\frac{v^2}{2g}$, $f$ is the friction factor\\ 

\begin{figure}[hb]
	\centering
	\includegraphics[width=0.7\linewidth]{"figs/problem 8"}
	\caption{Problem 8}
	\label{fig:problem-8}
\end{figure}

%\vfill
%\begin{center}{\bf END OF PAPER}\end{center}
\end{document}