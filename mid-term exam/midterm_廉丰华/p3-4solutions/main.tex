\documentclass{ctexart}
\author{}
\date{2016.11.17}
\title{mid-term exam}
\bibliographystyle{plain}
\newenvironment{keywords} {\\�ؼ��֣�\kaishu\zihao{-5}} {}
\usepackage{amsmath}
\usepackage{amssymb}
\usepackage[numbers,sort&compress,square,super]{natbib}
\usepackage{graphicx}
\usepackage{caption,subcaption}%����ͼ���ӱ���
\usepackage{asymptote}
%\usepackage{ctex}

\begin{document}
\maketitle

\begin{enumerate}
	\item
	let $t=0$
	\begin{align}
	\frac{dx}{dt}=x\\
	\frac{dy}{dt}=-y
	\end{align}
	$\Rightarrow$
	\begin{align}
	x=c_xe^t\\
	y=c_ye^{-t}
	\end{align}
	let $t=0$
	we have
	\begin{align}
	c_x=-1\\
	c_y=-1
	\end{align}
	so the streamline is
	\begin{align}
	xy=1
	\end{align}
	
	
		\begin{align}
		\frac{dx}{dt}=x+t\\
		\frac{dy}{dt}=-y+t
		\end{align}
		$\Rightarrow$
		\begin{align}
		x=c_xe^t-t-1\\
		y=c_ye^{-t}+t-1
		\end{align}
		use the condition of $x=-1,y=-1$ when $t=0$
		$\Rightarrow$
		\begin{align}
		x+y+2=0
		\end{align}
	
	
	\item 
	\begin{align}
	u=(a+1)e^t-1\\
	v=(b+1)e^t-1
	\end{align}
	
			$\Rightarrow$
			\begin{align}
			x=c_xe^t-t-1\\
			y=c_ye^{-t}-t-1
			\end{align}
			use the condition $x=a,y=b$ when $t=0$
			\begin{align}
			c_x=a+1\\
			c_y=b+1
			\end{align}
			so 
			\begin{align}
			u=(a+1)e^t-1\\
			v=(b+1)e^t-1
			\end{align}
			
			
%	\item
%	\begin{enumerate}
%		\item 
%		\begin{align}
%		\frac{\partial \rho \xi}{\partial t}+\mathbf{u}\cdot\nabla (\rho \xi)+\nabla \Theta=0
%		\end{align}
%		\item
%		the continuity equation:
%		\begin{align}
%		\frac{\partial \rho}{\partial t}+\frac{\rho }{den}
%		\end{align}
%	\end{enumerate}
	
	\item
	\begin{align}
	\frac{\partial (\rho A)}{\partial t}+\frac{\partial (\rho A u)}{\partial x}=0
	\end{align}
	
	\item
	\begin{align}
	\frac{\partial \rho}{\partial t}+\nabla \cdot (\rho \mathbf{u})=0
	\end{align}
	$\Rightarrow$
		\begin{align}
		\frac{\partial \rho}{\partial t}+\rho\nabla \cdot  \mathbf{u}+  \mathbf{u} \cdot\nabla \rho=0
		\end{align}
		using the incompressible condition
		\begin{align}
		\frac{d\rho}{dt}=\frac{\partial \rho}{\partial t}+  \mathbf{u} \cdot\nabla \rho=0
		\end{align}
		$\Rightarrow$
		\begin{align}
		\nabla \cdot \mathbf{u}=0
		\end{align}
\end{enumerate}


%\section{Mid-term exam}
%\begin{enumerate}
%	\item suppose a fluid element moves through the equation as follow,
%	\begin{align*}
%	x=2+0.01\sqrt{t^5}\\
%	y=2+0.01\sqrt{t^5}\\
%	z=2
%	\end{align*}
%	find the acceleration when the particle is at $x=8$
%	
%	\item
%	given a velocity field as follow,
%	\begin{align*}
%	u=yzt\\
%	v=zxt\\
%	w=0
%	\end{align*}
%	find the acceleration of fluid element at (2,5,3) when $t=10$
%	
%	
%	\item 
%	given a \textbf{incompressible} flow whose $x$ component of velocity is
%	\begin{align}
%	u=ax^2+by
%	\end{align}
%	where $a,b$ are constant, determine the $y$ component of velocity $v$. assume $y=0$ when $v=0$
%	
%	
%	\item
%	given a \textbf{incompressible} flow whose $x$ component of velocity is
%	\begin{align}
%	u=e^{-x}\cosh y +1
%	\end{align}
%	where $a,b$ are constant, determine the $y$ component of velocity $v$. assume $y=0$ when $v=0$
%	
%\end{enumerate}




\end{document}
