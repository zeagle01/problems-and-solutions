\documentclass[12pt]{article}
\usepackage{fancyhdr}
\usepackage{amsmath,amsfonts,enumerate}
\usepackage{color,graphicx}
\usepackage{bm}
\pagestyle{fancy}
%%%%%%%%%%%%%%%%%%%%%%%%%%%%%%%%%%%%%%%%%%%%%%%%%
% Do your customization here
%%%%%%%%%%%%%%%%%%%%%%%%%%%%%%%%%%%%%%%%%%%%%%%%%
\newcommand{\masunitnumber}{}
\newcommand{\examdate}{November 2016}
\newcommand{\academicyear}{2016-2017}
\newcommand{\semester}{I}
\newcommand{\coursename}{Transport Pheonomena}
\newcommand{\numberofhours}{2}

\newcommand{\ZZ}{\mathbb{Z}}
\newcommand{\CC}{\mathbb{C}}
\newcommand{\RR}{\mathbb{R}}
\newcommand{\FF}{\mathbb{F}}
%\DeclareMathOperator{\diam}{diam}
%%%%%%%%%%%%%%%%%%%%%%%%%%%%%%%%%%%%%%%%%%%%%%%%%
% Don't touch anything from here till instructions
% to candidates
%%%%%%%%%%%%%%%%%%%%%%%%%%%%%%%%%%%%%%%%%%%%%%%%%
\lhead{}
\rhead{}
\chead{{\bf SOUTHERN UNIVERSITY OF SCIENCE AND TECHNOLOGY}}
\lfoot{}
\rfoot{}
\cfoot{}
\begin{document}
\setlength{\headsep}{5truemm}
\setlength{\headheight}{14.5truemm}
\setlength{\voffset}{-0.45truein}
\renewcommand{\headrulewidth}{0.0pt}
\begin{center}
SEMESTER \semester\ EXAMINATION \academicyear
\end{center}
\begin{center}
{\bf \masunitnumber\ -- \coursename}
\end{center}
\vspace{20truemm}

\noindent \examdate\hspace{55truemm} TIME ALLOWED: \numberofhours\ HOURS

\vspace{19truemm}
\hrule
\vspace{19truemm}

\noindent\underline{INSTRUCTIONS TO CANDIDATES}
\vspace{8truemm}
%%%%%%%%%%%%%%%%%%%%%%%%%%%%%%%%%%%%%%%%%%%%%%%%%%%%%%
% Adjust your instructions here
%%%%%%%%%%%%%%%%%%%%%%%%%%%%%%%%%%%%%%%%%%%%%%%%%%%%%%
\begin{enumerate}
	\item This examination paper contains {\bf 8} questions.
	
	\item Answer all questions. 
	The marks for each question are indicated at the beginning of each question.
	
	
	\item This {\bf IS NOT an OPEN BOOK} exam.
	
	\item Candidates may use calculators. However, they should write down systematically the steps in the workings.
	
	\item Some formulas that might help.\\
	For laminar pipe flow,
	$u_{avg}=-\frac{1}{8\mu}\frac{\Delta p}{\Delta x}R^2$,$\Delta p$ is the pressure loss due to viscous effects, $R$ is the radius of pipe\\
	For pipe flow, $h_f=f\frac{L}{d}\frac{v^2}{2g}$, $f$ is the friction factor\\
	The acceleration of gravity $g=\mathrm{10m\cdot s^{-2}}$\\
\end{enumerate}


%%%%%%%%%%%%%%%%%%%%%%%%%%%%%%%%%%%%%%%%%%%%%%%%%
% leave this as it is
%%%%%%%%%%%%%%%%%%%%%%%%%%%%%%%%%%%%%%%%%%%%%%%%%
\newpage
\lhead{}
\rhead{\masunitnumber}
\chead{}
\lfoot{}
\cfoot{\thepage}
\rfoot{}
\setlength{\footskip}{45pt}
%%%%%%%%%%%%%%%%%%%%%%%%%%%%%%%%%%%%%%%%%%%%%%%%%%
% put your exam questions here
%%%%%%%%%%%%%%%%%%%%%%%%%%%%%%%%%%%%%%%%%%%%%%%%%%


\newpage
\paragraph{Problem 1.}\hfill (10 marks)\\
Two clean and parallel glass plates, separated by a gap of $b$ = 1.470 mm, are dipped in water. If coefficient of surface tension $\sigma$=0.0735 N/m , determine how high the water will rise.
(Assume the density of water $\rho_w=1 \times \mathrm{10^3 \, kg\cdot m^{-3}}$,the acceleration of gravity $g=\mathrm{10 \, m\cdot s^{-2}}$.)



\paragraph{Problem 2.}\hfill (14 marks)\\
The uniform beam in figure below, of size $L$ by $h$ by $b$ and with specific weight $\gamma_b$, floats exactly on its diagonal when a heavy uniform sphere is tied to the left corner, as shown. Show that this can happen only $(a)$ when $\gamma_b=\gamma/3$ and $(b)$ when the sphere has size
\begin{equation*}
D = \left[\frac{Lhb}{\mathrm{\pi}(\mathrm{SG} - 1)}\right]^{1/3}.
\end{equation*}
$Hint$: The specific weight ($\gamma$) is the weight per unit volume of a material. The specific gravity (SG) is defined as $\mathrm{SG} = \gamma_{\mathrm{sphere}}/\gamma$. The buoyancy of the beam, being a perfect triangle of displaced water, acts at $L/3$ from the left corner.

\begin{figure}[hb]
	\centering
	\includegraphics[width=0.5\linewidth]{"figs/problem 2A"}
	\caption{Problem 2}
	\label{fig:problem-2}
\end{figure}



\paragraph{Problem 3.}\hfill (14 marks)\\
Assume some 2-dimensional flow field satisfy
\begin{align*}
u=x+t\\
v=-y+t
\end{align*}
determine the streamline and pathline that is through point $(-1,-1)$,when $t=0$.
	


\paragraph{Problem 4.}\hfill (10 marks)\\
	Assume some flow in a tube is steady.the cross-section area,density,and velocity is $A(x),\rho(x),u(x)$ respectively,deduct the mass conservation equation.


\paragraph{Problem 5.}\hfill (12 marks)\\
Below is a picture which describes the siphon phenomenon: the water is siphoned from a large tank through a constant diameter hose. Assume water to be inviscid, incompressible and flow to be steady. (The acceleration of gravity $g=\mathrm{10 \, m\cdot s^{-2}}$.) Please determine:
\begin{enumerate}[(a)]
	\item
	velocity of water leaving (3) as a free jet
	\item
	water pressure in tube at (4)
\end{enumerate}
\begin{figure}[h]
	\centering
	\includegraphics[width=0.53\linewidth]{"figs/problem 5"}
	\caption{Problem 5}
	\label{fig:problem-5}
\end{figure}




\paragraph{Problem 6.}\hfill (12 marks)\\
At low velocities (laminar flow), the volume flow $Q$ through a small-bore tube is a function only of the tube radius $R$, the fluid viscosity $\mu$, and the pressure drop per unit tube length $\mathrm{d}p/\mathrm{d}x$. Please find an appropriate dimensionless relationship.


\newpage
\paragraph{Problem 7.}\hfill (14 marks)\\
Planar Couette flow is generated by placing a viscous fluid between two infinite parallel plates and moving one plate (say, the upper one) at a velocity U with respect to the other one. The plates are a distance h apart. Two immiscible viscous liquids are placed between the plates as shown in the diagram. The lower fluid layer has thickness d. Solve for the velocity distributions in the two fluids.
The viscosity of fluid 1 and fluid 2 is $\mu_1$ and $\mu_2$ respectively.
(for incompressible planar Couette flow,the mass conservation is $\nabla \cdot \textbf{u}=0$,and the momentum conservation is $\nabla^2 \textbf{u}=0$)

\begin{figure}[h]
	\centering
	\includegraphics[width=0.7\linewidth]{"figs/problem 7"}
	\caption{Problem 7}
	\label{fig:problem-7}
\end{figure}


\bigskip
\paragraph{Problem 8.}\hfill (14 marks)\\
An oil with $\rho=\mathrm{900 kg/m^3}$ and $\nu=\mathrm{0.0002 m^2/s}$ flows upward through an inclined pipe as shown in Fig.xxx. The pressure and elevation are known at sections 1 and 2, 10 m apart. Assuming steady laminar flow, (a) compute head loss $h_f$ between 1 and 2, and compute (b) V, (c) Re. Is the flow really laminar?(hint:$u_{avg}=-\frac{1}{8\mu}\frac{\rho g h_f}{\Delta l}R^2$)( The acceleration of gravity $g=\mathrm{10m\cdot s^{-2}}$)

\begin{figure}[h]
	\centering
	\includegraphics[width=0.5\linewidth]{"figs/problem 8"}
	\caption{Problem 8}
	\label{fig:problem-8}
\end{figure}

%\vfill
%\begin{center}{\bf END OF PAPER}\end{center}
\end{document}